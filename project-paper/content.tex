% status: 0
% chapter: TBD

\title{Big Data for Edge Computing}



\author{Janaki Mudvari Khatiwada}
\affiliation{%
  \institution{Indiana University}  
  \city{Bloomington} 
  \state{IN} 
  \postcode{47408}
  \country{USA}}
\email{janu.khatiwada@gmail.com}


% The default list of authors is too long for headers}
\renewcommand{\shortauthors}{J. M. Khatiwada}



\begin{abstract}
  This paper provides a simple demonstration of performing cloud computing in
  locally installed servers, raspberry pi clusters.
\end{abstract}

\keywords{e516, hid-sp18-415, raspberry pi,cluster, edge computing }


\maketitle`

\section{Introduction}
  Edge computing provides a simple solution of scalable, flexible, lightweight
  architecture and facilitates computation to the edge of the cloud rather than
  reaching for larger data centers and servers in the cloud. Raspberry Pi is a
  small, single-board device or computer that can be miodified into a platform
  for data
  storage, computation and deployment of applications near the edge of the cloud.
  They can
  be clustered to form a interoperable containers that can deploy applications near
  the edge. These devices are low cost (30 dollars for each pi), low power
  devices.
  
  In this project, we are going to build a 3 Raspberry Pi cluster to emulate a small
  scale edge computing platform. Then develop a
  restul web services application so that we can deploy them into the cluster.
  As a part of the
  project we are also going to deploy our application into Heroku so that we can
  benchmark whole process. Heroku is a Platform as a Service (PaaS) cloud
  application platform~\cite{www-heroku-com}.
  
\section{Conclusion}

Put here an conclusion. Conclusion and abstracts must not have any
citations in the section.


\begin{acks}

  The authors would like to thank Dr.~Gregor~von~Laszewski for his
  support and suggestions to write this paper.

\end{acks}

\bibliographystyle{ACM-Reference-Format}
\bibliography{report} 
