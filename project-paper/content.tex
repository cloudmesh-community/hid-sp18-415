% status: 60%
% chapter: TBD
\title{Simple Linear Regression API with Flask}



\author{Janaki Mudvari Khatiwada}
\affiliation{%
  \institution{Indiana University}  
  \city{Bloomington} 
  \state{IN} 
  \postcode{47408}
  \country{USA}}
\email{janu.khatiwada@gmail.com}


% The default list of authors is too long for headers}
\renewcommand{\shortauthors}{J. M. Khatiwada}



\begin{abstract}
  Python's flask application requires minimal application for application
  development. Flask make use of backend data as a resource to support an
  application running on the server. This project discusses development
  and deployment of REST api with python flask. This flask api runs linear
  regression model, an example of machine learning using python package
  SciKit--learn. For the purpose car data set from UCI machine learning
  data repository is being used. Before this a regression model for the
  data has to be build to make sure that this model best describes the 
  relationship between variables, in this case engine size and highway
  miles per gallon. Based on this model new data set can be tested to make 
  to make prediction. Then regression results is used as the endpoint 
  in flask api.
   
\end{abstract}

\keywords{e516, hid-sp18-415, Flask Api, Cloud Computing, Simple Linear Regression }


\maketitle`

\section{Introduction}
 
\section{Simple Linear Regression}
  Just to provide a brief overview of why simple linear regression is
  chosen for the project, a brief introduction about it would be
  appropriate .Linear Regression is a statistical data analysis tool
  which describes the effect of one variable also called independent
  variable on dependent variables. In case of simple linear regression
  there is only one dependent variable with one independent variable
  just like in multiple regression. Our purpose of running this project
  is to create a line of best fit model build a model that can be used
  for prediction. In this case prediction of mpg based on engine size
  of the vehicle. The line of best fit is described by the
  formula \[y = a + bx\] where a is coefficient or y intercept and b is
  slope for the line. In this case formula for line of best fit is
 \[hwy\_mpg = yintercept + slope.enginesize\] 
\section{Methods}
  Data selection, cleaning and preprocessing, building a local
  environment, building a linear regression model using SciKit-learn,
  designing a REST api with Flask and deploying  the api are the main
  processes of this project. They are explained below under each bullet
  points:
\begin{itemize}
    \item Automobile Data Set, \textit{https://archive.ics.uci.edu/ml/
          machine--learning-databases/autos/imports-85.data}, from UCI
          Machine learning data repository is selected. It was created
          in 19 May, 1987 by Jefferey C. Schlimmer \cite{uci-com}. The
          data set is multivariate with 25 attributes. Since the project
          is more focused on designing and deploying a simple linear
          regression api as a demonstration of implementing Flask service,
          only two attributes, `engine size' and `highway miles per gallon'
          are used. Rest of the attributes are dropped. The goal of the
          project is to build a best fit prediction model for highway mpg
          based on engine size. The data set have several missing instances
 as well which are removed as a clean up process. Cleaned data is stored in
 google Dropox as a csv   file so that it can be easily fetched into the
 application. It is located in \textit{https://www.dropbox.com/s/986566hudfytl8h/cardata.csv?dl=0}.
    \item Ubuntu 16.04 is the operating system used for whole project.
  A virtual environment pyenv is created in Ubuntu 16.04 where python 3
 environment is build. Since this project requires different python
 packages to fetch and manipulate data and run linear regression and
 then build a flask api. Using pip install package name, packages Flask,
 pandas, numpy, scipy and SciKit--learn are installed. Besides these
 packages, python packages matplotlib and seaborn are installed as they
 are required to create a regression plot and and plot line of best fit. 
    \item Next is building a model from the data set. This is done by
 running python's machine learning application. There are two types of
 machine learning, supervised and unsupervised learning. Supervised
 Machine learning is learning properties of data set (training data set)
 and applying them to the test data set. This machine learning algorithm
 is also called supervised learning. Regression problem in Scikit--learn
 is supervised machine learning. In this case predicting highway mpg for
 the new data set based on engine size is the regression problem.
    
    Regression analysis or building a best fit model for our data set is
   discussed in detail under section Analysis and Algorithm. Now the data
   set is splitted into two sets one used for defining an regression 
   equation that best fits the data. Once regression equation is defined
   that is value for slope and coefficient of constant is calculated.   
   Source codes are located in github repository \cite{}.
    \item The final process is writting a flask api which can be run to call
 endpoints performing different CRUD rest methods.
\end{itemize}    

\section{Analysis and Algorithms}
  
\section{Conclusion}

Put here an conclusion. Conclusion and abstracts must not have any
citations in the section.


\begin{acks}

  The authors would like to thank Dr.~Gregor~von~Laszewski for his
  support and suggestions to write this paper.

\end{acks}

\bibliographystyle{ACM-Reference-Format}
\bibliography{report} 
