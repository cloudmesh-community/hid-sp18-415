\section{Google Compute Engine}
\index{Google Compute Engine}

%%%%%%%%%%%%%%%%%%%%%%%%%%%%%%%%%%%%%%%%%%%%%%%%%%%%%%%%%%%%%%%%%%%%%%%%%%%%%%%%%
As an infrastructure as a service Google Compute Engine (GCE) provides scalable, 
high performance  virtual machines to their clients~\cite{hid-sp18-415-cloud-google}. 
Its virtual machines vary in CPU and RAM configurations  and Linux distributions 
depending on clients’ need. Network storage are attached to virtual machines are 
attached as persistent disks. Each of these disks’ size can be upto 64TB and 
they are automatically resized based on demands~\cite{hid-sp18-415-cloud-google}. 
This feature of GCE’s virtual machines makes it scalable and reliable. Another 
feature of GCE includes its global load balancing technology which allows 
distribution of multiple instances across different region~\cite{hid-sp18-415-cloud-google}. 
It provides a platform to connect with other virtual machines to form a cluster or 
connect to other data centers or other google services~\cite{hid-sp18-415-cloud-google}. 
It can be managed through RestFul API or command line interface or web console. 
It claims to be cost effective and environmentally friendly compared to its 
competitors like Amazon Web Services.
