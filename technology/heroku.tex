\section{Heroku}
\index{Heroku}
 
 Heroku, an open cloud platform provides developers a stage where they can develop
 and deploy their applications. It is a Platform as a service solution \cite{www-heroku}.
 Its applications are run in virtual containers called dynos and services are hosted by
 Amazon’s EC2 cloud computing platform  \cite{wikipedia-org}. Dynos support languages like
 Node, Python, Ruby, PHP, Scala, Clojure and Java. The applications or source code and
 their framework and dependencies can be written in any of the above specified languages
 as heroku supports them \cite{how-heroku-works}. Source code dependency vary according
 to the language being used. Applications are specified in a text file called
 ‘Procfile’ \cite{how-heroku-works}. Then these applications  are deployed remotely
 through git push. Besides Git, applications can be deployed to Heroku using GitHub
 integration, Dropbox Sync., and Heroku API \cite{how-heroku-works}. After applications
 are deployed, compilation of source code, their dependencies and necessary assets take
 place. The whole assembly of compilation is called slug \cite{how-heroku-works}.
 Then Heroku executes application by running command specified in Procfile.
 Commercial and business applications like Macy’s, Toyota use Heroku cloud platform
 because of its high scalability and its continuous deployment. PostgreSQL, MongoDB,
 Redis and Cloudant are common database choices of Heroku \cite{www-heroku}.
