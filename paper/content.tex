% status: 100 
% chapter: Paas

\title{Heroku Cloud Platform}


\author{Janaki Mudvari Khatiwada}
\affiliation{%
  \institution{Indiana University}
  \city{Bloomington} 
  \state{IN} 
  \postcode{47408}
  \country{USA}}
\email{janu.khatiwada@gmail.com}


% The default list of authors is too long for headers}
\renewcommand{\shortauthors}{J. M. Khatiwada}


\begin{abstract}
 Growing demands for internet of things and other smart devices increase
 need of web-based services and their applications. These applications and
 services are  used across various sectors such as trade, commerce, business,
 government, entertainment, research and development and day to day
 consumer needs. Cloud computing provides a medium where services and
 applications can be developed, deployed and managed over the web. These 
 services and appplications might need bigger (hardware) and different types
 (software) of infrastructures for data management, storage, backend data 
 access and data anlaysis. Cloud computing customers use cloud services
 for certain price or free if available so that they do not have to invest
 on the needed infrastructures. This means they deploy their applications and
 services while their needed infrastructures (hardware and software) and
 resources are hosted on other platforms. Heroku, a cloud based service provider
 provides developers a platform to implement
 locally developed applications to be hosted in a cloud based servers or
 virtual containers.It depends on Amazon Web Services for the infrastructure
 it needs to run its applications while developers do not have to worry about
 data management, servers and containers. This paper discusses features and
 applications of heroku cloud platform.
\end{abstract}

\keywords{e516, hid-sp18-415, Cloud Computing, Heroku, Dyno}


\maketitle



\section{Introduction}

 Cloud computing provides platforms, infrastructures and resources (databases)
 and their storage to the clients and developers. Based on type of services
 cloud provides they can be called \textit{Infrastructure as Service (Iaas),
 Plastform as Service ( Paas) or Software as Service (Saas)}. Heroku is a Paas
 provider that provides its services for application development, their management and
 deployment in a scalable manner. Scalable in the sense that if app gets 
 heavy web traffic, Heroku can increase the number of needed servers to manage
 and host the app. It is basically a platform as a service (PaaS) provider
 build on Amazon's Elastic Compute Cloud (EC2). ``It has a dynamic distributed
 runtime environment''~\cite{hid-sp18-415-www-devcenter-heroku}. It was founded
 in  2007 and has been
 in development since then~\cite{hid-sp18-415-www-wikipedia-org}.
 As a PaaS cloud service, heroku
 offers operating systems, hardware, servers and databases to application 
 developers. This certainly helps developers to save time and money as they do
 not have to focus on acquiring the necessary infrastructures like setting up 
 a server and installing software. While ruby, java, php, python and node are 
 the core languages supported by heroku, it supports variety 
 of other languages such as Scala and Clojure through 
 buildpacks~\cite{hid-sp18-415-www-wikipedia-org}.
 These buildpacks are third party add--ons. These add--ons provide features such as
 databases, continuous integration, performance monitoring and others as such. 
 They can be attached to the servers. It allows developers to locally build 
 source codes with the language that they prefer or are familiar with and then
 deploy them through git push. Applications are stateless that means they can 
 be reused. App deployments are managed through Git, which does version control
 and code management~\cite{hid-sp18-415-www-how-heroku-works}. Codes can also
 be ``deployed from GitHub or Continuous Integration (CI) 
 systems''~\cite{hid-sp18-415-www-heroku-com}.   
 Apps are run in virtual containers called Dynos. Number of dynos or virtual 
 servers are preset in the application but they can be instantly scaled to 
 support the needs~\cite{hid-sp18-415-www-how-heroku-works}. Redis and Postgres
 or PostgreSQL are database services of heroku available through 
 add--ons~\cite{hid-sp18-415-www-heroku-com}.

\subsection{Continuous Integration System}

 Heroku Continuous Integration (CI) system is a ``visual, low configuration
 test runner that runs app in automatically created disposable 
 heroku apps~\cite{hid-sp18-415-www-heroku-ci}.


\subsection{Dynos}

 Heroku dynos, building blocks of heroku app, are virtual linux containers or
 virtual servers which run developers specified source codes once they are 
 deployed~\cite{hid-sp18-415-www-heroku-dynos}. These containers provide necessary
 environment
 to run an application. Unless specified, app will start running with one dyno 
 with  web request but they can be scaled depending upon the volume of the 
 application traffic~\cite{hid-sp18-415-www-heroku-dynos}. Number of Dynos, their RAM and 
 CPU can be 
 configured based on application's needs. From Heroku Dashboard or Command Line
 Interface (CLI) these dynos can be scaled manually up and down or also can be 
 autoscaled~\cite{hid-sp18-415-www-heroku-dynos}. Each dyno has a set of 
 ``unique properties and performance characterstics''~\cite{hid-sp18-415-www-heroku-dynos}, 
 based on which they are categorized as ``Free, Hobby, Performance and Standard 
 dynos''~\cite{hid-sp18-415-www-devcenter-dynos}. Heroku Enterprise has private dynos.

 Based on the configurations they are further
 categorized as Web dyno, Worker or One--off dynos. Dynos that receive HTTP
 requests are web dynos and are of web process type in procfile. Worker dynos
 are responsible for background jobs and queuing~\cite{hid-sp18-415-www-devcenter-dynos}.
 ``Temporary dynos called One-off dynos with their input/output can be created 
 and attached to local terminal''~\cite{hid-sp18-415-www-how-heroku-works}. They will be 
 terminated due to inactivity or session termination.
 The dyno manager is responsible for keeping
 dynos run automatically, that is restart and stop~\cite{hid-sp18-415-www-devcenter-dynos}. 
 Dyno manager manages all dynos across all applications on Heroku. Its job is 
to make sure dynos are running and cycled at least once a day, detects fault
 in applications or hardware, awaken sleeping application when it receives http
requests~\cite{hid-sp18-415-www-how-heroku-works}. 


 \section{Architecture}

 Heroku's architecture on the developer's side include, source codes, a list of
 dependencies listed in text file called requirements.txt and a Procfile. Its
 architecture is stateless and scalable both horizontally and vertically.
 Vertical scaling is done by increasing the memory of running dyno and
 horizontal scaling is increasing the number of dynos as required by
 the traffic~\cite{hid-sp18-415-www-python-heroku-com}. Scaling can be done either
 through commmand line interface (CLI) by writing a command or simply
 by sliding a bar on Heroku dashboard. 

 A Procfile is a text file with 
 process command to be used to start running the code. Process commands might
 be web command if a developer wants to run application in the web or queue
 command to include app in queue before deploying~\cite{hid-sp18-415-www-how-heroku-works}.
 Procfile basically specifies the process to be executed.
 Source code can be presented in developer's preferred language.
 Next step, application deployment to Heroku is done preferably using Git
 as remote repository associates with local git  once app is created in Heroku.
 But locally developed applications can also be send to Heroku using Github,
 Dropbox or through API~\cite{hid-sp18-415-www-how-heroku-works}.Heroku Platform starts 
 building source application once it receives application source codes. This 
 process also called buildpacks which retrieves app dependencies as specified 
 in the source and generates output in the form of compiled code. Then along
 with language and framework, application source code, retrieved dependencies
 and compiled code are assembled for execution. This bundled assembly is also
 called slug.

 Heroku allows to develop and deploy applications in different environments.
 This makes heroku portable. Here,environment means language and operating
 systems. Configuration variables or config vars are environment variables
 that have specific information about environment differences. They are
 separately located from slug. They are dynamic, so can be changed or
 customized independently from source code~\cite{hid-sp18-415-www-how-heroku-works}.

 Heroku executes applications in a dyno, proloaded with slug and configuration
 variables, by running a command specified in Procfile. Developers can specify
 number of dynos needed to run the app in the procfile. In other words,
 specification of dynos in the procfile allows application scaling. When new
 version of applications are deployed all of the currently executing dynos are
 killed and newer ones replace them preserving the existing dyno
 formation~\cite{hid-sp18-415-www-how-heroku-works}. HTTP routers distribute application
 requests across all running web dynos. 


\section{Python API on Heroku}

 Python is one of the in-built languages supported on Heroku cloud platform.
 APIs and apps build on any version of python can be deployed to heroku.
 Following steps are referenced from Heroku documentation. 

 To create a simple web API using Flask framework, initiate the process
 by creating a git repository then under this directory specify needed
 dependencies like python version, Flask, and Gunicorn. Gunicorn
 enables concurrent  http requests. If these
 dependencies are not available locally, first create a virtualenv or
 pyenv then perform pip install--dependency name (e.g., gunicorn). 
 These dependencies are
 then stored in requirements.txt file by running  command pip install 
 requirements.txt. To make sure the dependencies do
 not get changed while deploying app in Heroku, we can run pip freeze
 requirements.txt command. Then create a simple web API using flask under the same
 directory. Next step is creating a \textit{Procfile}. Procfile is a text file
 that has information
 about the application process. Process might be web or queue or others depending on 
 what we want to run in our API. Now initialize local git repository with command
 git init and git add, git commit and git push add these locally created files 
 into git repository.

 Command, Heroku create will create \textit{herokuapp.com}. Now locally created API
 can be deployed to Heroku by command git push Heroku master. Curl command
 can be issued to test newly created API or it can be tested in the web
 browser as well.


\section{Heroku Use-Cases}

 Heroku seems to be affordable to a start-up business, since it is a platform
 as a service cloud system. Companies do not have to worry about managing
 servers, upgrading the system to make it scalable and dealing with security
 issues. These add up to be an extra costs for businesses. 

 Ivisalign is a user-end app for Align Technology Inc., which is a provider of
 invisible orthodontic products to the customers. ``Along with invisalign.com app,
 other consumer--facing apps of Align Inc., are run on
 Heroku''~\cite{hid-sp18-415-www-customers-heroku-com}. This allows developers to focus
 on apps instead of managing the infrastructures, so that the app can be quickly
 updated on a daily basis with minimal changes in the source code. Heroku also
 allows to scale up HTTP requests and process simply through dashboard or a
 single command in heroku CLI. 

 Apartment List is another app built on Heroku and available for iPhone and
 Android. Apartment List Inc., allows customers to search apartments by
 connecting them to a consolidated listing of available apartments in
 their area~\cite{hid-sp18-415-www-customers-heroku-com}. This app is location based
 and requires real time results. It serves customers by helping them search
 neighborhoods based on street address and displays an entire inventory for
 every neighborhood in real time~\cite{hid-sp18-415-www-customers-heroku-com}.
 `` Apartment List has a team that works on their native app, a team
 that works on their web app and a team that works on the in-house rental
 specialist app  all running on Heroku''~\cite{hid-sp18-415-www-customers-heroku-com}.

 Another example this paper likes to showcase is Olga Tennison Autism Research
 Center (OATRC), Australia. This center provides a platform for autism related
 research activities, evidence--based intervention programs, training
 opportunities and collaboration between stakeholders in Australia and
 overseas~\cite{hid-sp18-415-www-customers-heroku-com}. Based on comprehensive research and
 the data gathered from trained nurses to study behaviors in children aged 
 between 12 to 24 months, predictive algorithm for Autism Syndrome Detection (ASD)
 was developed into ASDetect app. This app was launched in 
 2016~\cite{hid-sp18-415-www-customers-heroku-com} and runs on Heroku platform. Now parents
 from all around the world can track observations of behavior of their child.
 This app allows parents to look for risk factors and needed guidance.

 Toyota Motor Europe, CiTriX, Macy's Inc.,Lynx Internet of Grilled Things,
 Dubsmash, Dollar General are few of many more companies running their apps
 on Heroku.


\section{Limitations}

 Free dyno has 512 MB of RAM and as the name suggests it is free. Free dyno
 sleeps for 6 hours~\cite{hid-sp18-415-www-how-heroku-works}. For others like performance
 and standard dynos RAM is higher but they may not be free. If dynos need to
 be scaled both in numbers and RAM size there is extra charge for
 that~\cite{hid-sp18-415-www-how-heroku-works}. HTTP requests on heroku has 30 seconds
 window for some kind of response. Once it responds a 50 seconds window is
 rolled up in both client side and server side~\cite{hid-sp18-415-www-devcenter-herokulimits}.

\section{Conclusion}

 Heroku cloud platform, a Paas solution allows developers focus on building 
 applications rather than spending time, energy and money on infrastructures
 needed for application. While apps are developed locally they are deployed
 through github. And rest of the processes will be managed by Heroku. Its
 multi language support makes it portable. It provides free server called dyno
 for minimal application. However, as per higher traffic dynos can be scaled but
 developers might have to pay for increased number of dynos. Companies can run
 their apps in heroku without having to worry about investing in infrastructures
 and managing them.


\begin{acks}

  The authors would like to thank Dr.~Gregor~von~Laszewski for his
  support and suggestions to write this paper.

\end{acks}

\bibliographystyle{ACM-Reference-Format}
\bibliography{report} 

